\documentclass[14pt,aspectratio=169]{beamer}

% Used packages
\usepackage{amsthm}
\usepackage{tikz,tcolorbox}
\usepackage[object=vectorian]{pgfornament}
\usepackage[active,tightpage]{preview}
\usepackage[T1]{fontenc}
\usepackage{ebgaramond}
\usepackage{pgfplots}  % For plotting graphs
\usepackage{pgf-pie}   % For pie charts

% Package options
\PassOptionsToPackage{dvipsnames,svgnames}{xcolor}
\usetikzlibrary{shapes.geometric,calc}
\definecolor{fondpaille}{cmyk}{0,0,0.1,0}
\definecolor{Maroon}{cmyk}{0,2,0.3,0}
\definecolor{BrickRed}{rgb}{0.8, 0.25, 0.33}  % Define BrickRed color

% Slide Tweaks
\usetheme{Copenhagen}
\usecolortheme{beaver}
\setbeamertemplate{navigation symbols}{}
\setbeamertemplate{headline}{}

% Each slide corner ornament
\newcommand{\eachpageornament}{%
\begin{tikzpicture}[remember picture, overlay, color=BrickRed]
\node[anchor=north west] at (current page.north west){%
\pgfornament[width=2cm]{63}};
\node[anchor=north east] at (current page.north east){%
\pgfornament[width=2cm,symmetry=v]{63}};
\node[anchor=south west] at (current page.south west){%
\pgfornament[width=2cm,symmetry=h]{63}};
\node[anchor=south east] at (current page.south east){%
\pgfornament[width=2cm,symmetry=c]{63}};
\end{tikzpicture}
}

% Title and author 
\title{Data Visualization Presentation}
\author{Miskatul Anwar}
\date{September 2024}

\begin{document}
\pagecolor{fondpaille}
\color{Maroon}
\setlength{\PreviewBorder}{1em}

% Title slide
\begin{frame}
    \titlepage
    \eachpageornament
\end{frame}

% Importance of Data Visualization
\begin{frame}{Importance of Data Visualization}
    \eachpageornament
    \begin{itemize}
        \item Simplifies complex data.
        \item Reveals patterns and trends.
        \item Helps in decision making.
    \end{itemize}
\end{frame}

% Bar Plot Slide
\begin{frame}{Bar Plot}
    \eachpageornament
    \begin{itemize}
        \item Shows categorical data with rectangular bars.
        \item Useful for comparing quantities across categories.
    \end{itemize}
    \centering
    \begin{tikzpicture}
        \begin{axis}[
            ybar,
            symbolic x coords={A, B, C, D},
            xtick=data,
            ylabel={Values},
            xlabel={Categories},
            nodes near coords,
            width=0.7\textwidth,
            height=0.5\textwidth
        ]
        \addplot coordinates {(A,10) (B,15) (C,7) (D,12)};
        \end{axis}
    \end{tikzpicture}
\end{frame}

% Pie Chart Slide
\begin{frame}{Pie Chart}
    \eachpageornament
    \begin{itemize}
        \item Represents data in a circular graph.
        \item Useful for showing proportions.
    \end{itemize}
    \centering
    \begin{tikzpicture}
        \pie[radius=2, text=legend, color={red!40, blue!40, green!40, yellow!40}]
        {20/A, 30/B, 25/C, 25/D}
    \end{tikzpicture}
\end{frame}

% Line Plot Slide
\begin{frame}{Line Plot}
    \eachpageornament
    \begin{itemize}
        \item Represents data points connected by straight lines.
        \item Ideal for showing trends over time.
    \end{itemize}
    \centering
    \begin{tikzpicture}
        \begin{axis}[
            xlabel={Time},
            ylabel={Value},
            width=0.7\textwidth,
            height=0.5\textwidth
        ]
        \addplot coordinates {(1,10) (2,15) (3,9) (4,12)};
        \end{axis}
    \end{tikzpicture}
\end{frame}

% Heatmap Slide
\begin{frame}{Heatmap}
    \eachpageornament
    \begin{itemize}
        \item Visualizes data in matrix format.
        \item Useful for highlighting correlations.
    \end{itemize}
    \centering
    \begin{tikzpicture}
        \begin{axis}[
            colormap name=viridis,
            colorbar,
            width=0.8\textwidth,
            height=0.5\textwidth,
            mesh/cols=3  % Specify the number of columns
            ]
            % Ensure there are at least 2 rows and 2 columns for the matrix plot
            \addplot [matrix plot*,point meta=explicit] coordinates {
                (1,1) [2] (1,2) [3] (1,3) [1]
                (2,1) [4] (2,2) [5] (2,3) [3]
                (3,1) [1] (3,2) [2] (3,3) [4]
            };
        \end{axis}
    \end{tikzpicture}
\end{frame}

% Thank You Slide
\begin{frame}{Thank You}
    \eachpageornament
    \centering
    \Huge{Thank You!}
\end{frame}

\end{document}
